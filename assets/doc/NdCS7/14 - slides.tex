\documentclass[french, aspectratio=169, 14pt, handout]{beamer}
% [handout] mode ignores the step by step presentation
% [aspectratio=169]

% \usepackage{lmodern} % make sure the following line does not produce blurry texts
\usepackage[T1]{fontenc} % encode non English letters properly
% \usepackage[utf8]{inputenc}
% \usepackage{babel}
\usepackage{CJKutf8}


\setbeamertemplate{navigation symbols}{} % gets rid of navigation symbols at the bottom
\setbeamertemplate{itemize item}{$\circ$} % change default item/subitem/subsubitem symbols
%\setbeamersize{text margin left=5pt,text margin right=5pt}
\setbeamertemplate{itemize subitem}{$\circ$}

% overlay specification
%\setbeamercovered{transparent}
% \beamerdefaultoverlayspecification{<+->}

\usepackage{amsmath, amssymb}
%\usepackage{amsmath,amssymb,amsthm,amscd,tikz-cd,xfrac,faktor} % Need to add [fragile] after \begin{frame} to ensure that tikz-cd works
%\usepackage{mathrsfs}
%\usepackage[all]{xy}
\usepackage{setspace}
\usepackage[normalem]{ulem} % strikethrough line
\newcommand{\red}[1]{\textcolor{red}{#1}} % something more static than \alert
\newcommand{\blue}[1]{\textcolor{blue}{#1}} % something more static than \alert


\title[NdCS7-14]{C'est remboursable ?} % The short title appears at the bottom of every slide, the full title is only on the title page

\author{Séance 14} % Your name

\date{\today} % Date, can be changed to a custom date


\begin{document}

\begin{CJK*}{UTF8}{gkai}
% fonts: gbsn/gkai

\begin{frame}
\titlepage % Print the title page as the first slide
\end{frame}

\begin{frame}{上周作业}
\begin{itemize}
	\item 拼写及读音:lav\red{er},过去分词~\alt<+(1)->{lav\red{é}}{[lavé/lave]}, \pause 现在时~il/elle lav\red{e} \pause \\ 读音:laver = lavé $\neq$ lave \pause
    \item[]
    \item avoir + 过去分词
    \item 过去分词本身具有“已经……的”的含义,像个形容词, \pause \\ \red{助动词是~être 时,和主语进行性、数搭配}。
\end{itemize}
\end{frame}

\begin{frame}{位移动词的助动词:不及物用~être,及物用~avoir}
动作有受体时为“及物”,无受体时为“不及物”。 \pause
\begin{itemize}
    \item sortir \\
	他们-出去了。出去的是“他们”。\red{Ils} sont sorti\red{s}. \\
    他们-拿出了-一些设备。出来的是“设备”。\\ \hspace{170px} Ils ont sorti quelques appareils… \pause
    \item passer \\
    他们-经过了。经过的是“他们”。\red{Ils} sont passé\red{s}. \\
    他们-度过了-一小时。度过的是“1h”。Ils ont passé 1h… \pause
    \item 词典里分别标记为: \\
    verbe intransitif (v. intr.) 不及物动词 \\
    verbe transitif (v. tr.) 及物动词
\end{itemize}
\end{frame}

\begin{frame}{自反动词/代动词}
当动作的受体是主语或是主语的一部分时,为“自反动词”。 \pause
\begin{itemize}
    \item laver 洗,se laver 洗自己=洗澡 \\
	她洗了件衬衫。洗了的不是“她”。Elle a lavé une chemise. \\
	她洗澡。洗了的是“她自己”。Elle \red{s'}est lavé\red{e}. \pause
    \item 词典里分别标记为: \\
    verbe transitif (v. tr.) 及物\\
    verbe pronominal (v. pron.) 自反动词/代词式动词 \pause
\end{itemize}
过去分词:\blue{如果“已经……的”东西在动词前,才需搭配。} \pause
\begin{itemize}
    \item 她洗了手。洗了的是“手”。Elle s'est lavé les mains.
\end{itemize}
\end{frame}

\begin{frame}{否定词~pas 的位置}
\begin{itemize}
	\item (词典中)动词原形:ne pas faire qqch.  \pause
	\item \red{变位后}:否定词~pas 紧接\red{变位的动词}\red{之后}。\pause
	\item ne pas oublier (忘记) le ticket \pause \\
	现在时:\alt<+(1)->{Vous \only<+(1)->{n'}oubliez \red{pas}}{Vous oubliez} le ticket.  \pause \\
	过去时:\alt<+(1)->{Vous \only<+(1)->{n'}avez \red{pas} oublié}{Vous avez oublié} le ticket.  \pause
	\item ne pas se comprendre 不互相理解 \pause \\
	现在时:\alt<+(1)->{Ils \only<+(1)->{ne} se comprennent \red{pas}}{Ils se comprennent}.  \pause \\
	过去时:\alt<+(1)->{Ils \only<+(1)->{ne} se sont \red{pas} compris}{Ils se sont compris}.  \pause
	\item 口语中,ne 经常省略。
\end{itemize}
\end{frame}

\begin{frame}[c]{I. Révision \& nouveau vocabulaire 复习和新词汇}
\begin{itemize}
	\item Faire des achats 购物(颜色、服装) \\ \visible<+(1)>{\blue{——名词和形容词的性、数配合,quel/quelle/quels/quelles}}
	\item Au marché 在集市(食品名称、店名) \\ \visible<+(1)>{\blue{——冠词(定冠词、不定冠词、部分冠词,冠词的缩合)}}
	\item Le transport 交通(交通工具)\\ Se diriger 找路(直走、左转、右转、走哪条路) \\ \visible<+(1)>{\blue{——à+le=au,à+les=aux(à l' 和~à la 不变),命令式}}
	\item Le dimanche matin 周日的上午(日常动作) \\ \visible<+(1)>{\blue{——自反动词的现在时}}
	\item Le passé 过去(描述过去发生的事) \\ \visible<+(1)>{\blue{—— le passé composé 复合过去时}}
\end{itemize}
\end{frame}

\begin{frame}{1. Je voudrais + 名词/动词原形}
用于礼貌地表达「我想要」。
\begin{itemize}
    \item 我想要一根法棍。 Je voudrais une baguette. \pause \\
    我想要\red{一些}土豆。 \pause Je voudrais \pause \red{des} patates. \pause \\
    我想要\red{一些}肉。 \pause Je voudrais \pause \red{de la} viande.  \pause \\
    \item 我想要~800 克番茄。 \pause Je voudrais \pause \red{800 grammes} \pause \red{de} \pause tomates. \pause \\
    再多一点。\pause Un peu plus. \pause \\
    再少一点。\pause Un peu moins. \pause \\
    可以了!\pause C’est bon/ça va. \pause
    \item 我想要一件套衫。 \pause Je voudrais un pull.  \pause \\
    我想要一件\visible<.(3)->{蓝色}衬衫。 \pause Je voudrais \red{une} chemise \pause bleu\pause{\red{e}}.  \pause
	\item 我想\red{买}一件套衫。 \pause Je voudrais \pause \red{acheter} \pause un pull. 
\end{itemize}
\end{frame}

\begin{frame}{2. il y a + 时长,「(多久)以前」}
\begin{itemize}
	\item 买~acheter $\rightarrow$ 买了~\pause avoir acheté,变位变~avoir。\pause  \\
	我三天前买了这件套衫。 \pause \\ J'ai acheté \pause ce pull \pause \red{il y a} \pause 3 jours.  \pause \\
	我们一周前买了这件套衫。\pause \\ Nous avons acheté \pause ce pull \pause \red{il y a} \pause une semaine.  \pause
	\item 完成~finir $\rightarrow$ 完成了~\pause avoir fini \pause \\
	Alexandre 两小时前做完了作业。 \pause \\ Alexandre a fini \pause \alert<+(1)>{les} devoirs \pause \red{il y a} \pause  2 heures.  \pause \\
	她\alert<.(5)->{还}\alert<.(3)->{没}做完作业。\pause \\ Elle \only<.(6)->{n'}a \visible<.(3)->{\alt<.(5)->{\red{pas encore}}{\red{pas}}} fini les devoirs.
\end{itemize}
\end{frame}

\begin{frame}{3. 冠词:圈出正确的选项。}
\begin{itemize}
	\item 您有\red{您的}票吗? Vous avez votre ticket ?  \pause \\
	\hspace{80px} Vous avez \alt<+(1)->{\red{le}}{[le/un/du]} ticket ?  \pause \\
	我有\red{我的}票。 J'ai mon ticket.  \pause \\
	\hspace{70px} J’ai \alt<+(1)->{\red{le}}{[le/un/du]} ticket.  \pause
	\item 您有银行卡吗? Vous avez \alt<+(1)->{\red{une/la}}{[la/une/de la]} carte bleue ? \pause 意思不同,取决于问“卡是否存在”还是“有没有某张特定的卡”。 \pause \\
	我有张银行卡。 J’ai \alt<+(1)->{\red{une}}{[la/une/de la]} carte bleue.  \pause \\
    我有\red{我的}银行卡(有并且带了)。 J'ai ma carte bleue.  \pause \\
    \hspace{180px} J’ai \alt<+(1)->{\red{la}}{[la/une/de la]} carte bleue. \pause \\
	我\visible<+(1)->{\red{没}}有\red{我的}银行卡\visible<.(1)->{(没带)}。 \alt<+(2)->{Je \red{n'}ai}{J'ai} \visible<+->{\red{pas}} ma/la carte bleue. 
\end{itemize}
\end{frame}

\begin{frame}{3. 冠词:圈出正确的选项。}
\begin{itemize}
	\item 您有钱吗?Vous avez \alt<+(1)->{\red{de l'}}{[le/un/du/de l’]} argent (m.) ? \pause \\
	您有零钱吗? Vous avez \alt<+(1)->{\red{de la}}{[la/une/de la]} monnaie (f.)? \pause \\
	您有硬币吗?Vous avez \alt<+(1)->{\red{des}}{[les/des]} pièces (f. pl.)?
\end{itemize}
\end{frame}

\begin{frame}{4. perdre 「丢失」,过去分词~perdu}
\begin{itemize}
	\item 我昨天丢了票。Hier, \pause j'ai perdu \pause \red{mon/le} ticket.  \pause
	\item 我左转了,我右转了, \pause \alt<+(1)->{J'ai tourné à gauche/à droite, }{tourner à gauche/à droite} \pause \\
	然后我迷路了(丢了自己)。puis/après, je \alt<+(1)->{\alt<+(1)->{\red{me suis perdu}}{[m'être perdu]}}{[se perdre].}.	\pause \\
	\hspace{155px} (si je suis \red{女的}) je \red{me} suis perdu\red{e}. \pause
	\item 你们迷路了吗? Vous \alt<+(1)->{\alt<+(1)->{\red{vous êtes perdus}}{[vous être perdu]}}{[se perdre]} ? \pause \\
	\hspace{60px} Vous \red{vous} êtes perdu\red{es} ? (si vous êtes \red{toutes 女的})
\end{itemize}
\end{frame}

\begin{frame}{5. garder 保留,保存(东西)}
\begin{itemize}
	\item 保留好您的小票! Gard\red{ez} votre ticket.  \pause
	\item 我保留了小票。 J'ai gardé mon ticket.  \pause
	\item 我们没保留小票。 Nous \only<+(1)->{\red{n'}}avons \red{pas} gardé le ticket/les tickets. 
\end{itemize}
\end{frame}

\begin{frame}{6. oublier 忘记(+ 名词)}
\begin{itemize}
    \item 忘了~avoir oublié \pause
    \item 昨天,我忘了我的包。 \pause \\ Hier, j'ai oublié mon sac. \pause
	\item \red{不要忘了}包! \pause \\ \alt<+(1)->{\red{N'oublie pas}}{\red{oublie\sout{s} pas}} ton sac.  \pause \\
	\red{不要忘了}您的小票! \pause \\ \red{N'oubliez pas} \pause votre ticket.  \pause
\end{itemize}
\end{frame}

\begin{frame}{6. oublier 忘记(+ de+动词原形)}
\begin{itemize}
	\item 今天上午,我忘了运动。 \pause \\ Ce matin, j'ai oublié \pause \red{de} \pause \red{faire} du sport. \pause
	\item 不要忘了运动!\\ N'oublie pas/N'oubliez pas \pause \red{de} \pause \red{faire} du sport !  \pause \\ 
	不要忘了保留小票!\\ N'oublie pas/N'oubliez pas \pause \red{de} \pause \red{garder} votre ticket. 
\end{itemize}
\end{frame}

\begin{frame}{7. 最近将来时:aller + 动词原形}
\begin{itemize}
	\item 还不错,OK,可以的,行的。
	\visible<+(1)->{$\rightarrow$  会OK的,会行的。} \\
	Ça va.  \visible<+(1)->{$\rightarrow$}  \visible<+(1)->{Ça va aller.}  \\
	Ça marche.  \visible<+(1)->{$\rightarrow$} \visible<+(1)->{Ça va marcher}.
\end{itemize}
\end{frame}

\begin{frame}{8. pouvoir + 动词原形,表示「可以……」}
\begin{itemize}
	\item 我可以吗?(想做某件事、征求他人同意时) \pause \\ Je peux ? \pause
	\item 我可以帮您吗? \pause \\ Je peux \pause vous aider ? \pause
	\item 我可以试试(试穿/试用)它吗? \pause \\ Je peux \pause l’essayer ? \pause
	\item[]
	\item 您可以直走然后右转。 \pause \\ Vous pouvez \pause continuer tout droit, puis \pause tourner à droite.
\end{itemize}
\end{frame}

\begin{frame}{8. pouvoir + 动词原形,表示「可以……」}
\begin{itemize}
	\item 您可以早点儿起床吗? \\ Vous pouvez \visible<+(1)->{\alt<+(1)->{\red{vous} lever}{se lever}} plus tôt ?  \pause
	\item 咱不能工作却不休息。	\pause \\ On \alt<+(4)->{ne peut}{peut} \visible<+->{\red{pas}} \visible<+->{travailler} sans \visible<+->{se reposer}. \pause
	\item 可以吗?行? \\
	Ça va ?  $\rightarrow$  \pause Ça peut aller ? \pause \\
	Ça marche ? $\rightarrow$ \pause Ça peut marcher ? 
\end{itemize}
\end{frame}

\begin{frame}{9. 退款/换货}
\begin{itemize}
	\item \alert<+>{我可以}给您退款。 \\ \visible<.(1)->{Je peux} vous rembourser.  \pause
	\item \alert<+>{您可以}换成另一件套衫。 \\ \visible<.(1)->{Vous pouvez} changer pour un autre pull.  \pause
	\item[]
	\item \red{退款:rembourser} \pause \\
	\item \red{换:échanger/changer}
\end{itemize}
\end{frame}

\begin{frame}{9. 退款/换货}
\begin{itemize}
	\item \red{退款(动词):rembourser} \pause \\
	退款(名词):rembourse\red{ment} (m.) \pause \\
	可退款的:rembours\red{able} \pause \\
	(人)获得退款(被给予退款):être rembours\red{é}.  \pause \\
	分词如同形容词,要性数配合:\hspace{0px} Il est rembours\red{é}.  \pause \\
	\hspace{200px} Elle est rembours\red{ée}.  \pause \\
	\hspace{200px} Ils sont rembours\red{és.}  \pause \\
	\hspace{190px} Elles sont rembours\red{ées.}
\end{itemize}
\end{frame}

\begin{frame}{9. 退款/换货}
\begin{itemize}
    \item % 把~A 换成~B:échanger/changer A contre/pour B \pause \\
	把这件衬衫\red{换掉}:\pause \red{changer} cette chemise \pause \\
	\red{换成}那边的衬衫:\pause \red{changer pour} la chemise là-bas \pause \\
	\hspace{160px} \red{pour} 表示目的 \pause \\
	\item 交换:échange (m.) \pause
	\item 可换的:échange\red{able} \pause
	\item[]
	\item 可退换的:\pause rembousable et échangeable
\end{itemize}
\end{frame}

\begin{frame}{II. Dans une grande surface 在一家大商场}
\begin{itemize}
	\item J'ai acheté ce pull \alert<+>{il y a deux jours}, mais il est vraiment \alert<+>{trop} petit. Est-ce que \alert<+>{je peux le changer} ?
	\item Oui, madame, vous avez votre \alert<+>{ticket de caisse} ?
	\item Ah non, \alert<+>{je ne sais pas} où il est, \alert<+>{je crois} que \alert<+>{je l’ai perdu} ! \pause
	\item Madame, \alert<+>{on ne peut pas changer} un article \alt<+>{\red{sans} (没有)}{sans} le ticket de caisse.  \pause
	\item J’ai peut-être encore le \alert<+>{ticket de carte bleue}. Est-ce que \alert<+>{ça peut aller} ? \pause
	\item D’accord, ça pourra aller...  \pause Mais \alert<+>{n'oubliez pas de garder} votre ticket de caisse, \alt<+>{\red{si} (如果)}{si} vous \alert<+>{souhaitez} \alert<+>{faire un échange} ou \alert<+>{être remboursée} !  \\
	\visible<.(-2)>{souhaiter + 动词原形 「想要做某事」}
\end{itemize}
\end{frame}

\begin{frame}{}
\huge{\centerline{C'est terminé. Bonnes vacances !}}
\end{frame}

\end{CJK*}

\end{document}
